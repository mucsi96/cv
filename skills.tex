%-------------------------------------------------------------------------------
%	SECTION TITLE
%-------------------------------------------------------------------------------
\cvsection{Skills}


%-------------------------------------------------------------------------------
%	CONTENT
%-------------------------------------------------------------------------------
\begin{cvskills}

%---------------------------------------------------------
  \cvskill
    {Front-end} % Category
    {React, Redux, MobX, HTML5, CSS3, LESS, SASS, CSS Modules, Webpack, Jest, Babel (8 years in total)} % Skills
%---------------------------------------------------------
  \cvskill
    {Back-end} % Category
    {Node.js, Koa, Express, REST API, SQL, GraphQL, MongoDB, Redis, RabbitMQ (3 years in total)} % Skills
%---------------------------------------------------------
  \cvskill
    {Programming} % Category
    {JavaScript ES6+, TypeScript, Java} % Skills
%---------------------------------------------------------
  \cvskill
    {DevOps} % Category
    {GitLab, GitHub, Jenkins, TravisCI, Docker, Kubernetes, AWS, Google Cloud Platform} % Skills
%---------------------------------------------------------
  \cvskill
    {Languages} % Category
    {Hungarian, English, German (beginner level)} % Skills

%---------------------------------------------------------
\end{cvskills}
